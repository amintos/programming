% -*- coding: utf-8; -*-
% vim: set fileencoding=utf-8 :
\documentclass[english,submission]{programming}

\usepackage[backend=biber]{biblatex}
\addbibresource{example.bib}

\usepackage{multicol}

\lstdefinelanguage[programming]{TeX}[AlLaTeX]{TeX}{%
  deletetexcs={title,author,bibliography},%
  deletekeywords={tabular},
  morekeywords={abstract},%
  moretexcs={chapter},%
  moretexcs=[2]{title,author,subtitle,keywords,maketitle,titlerunning,authorinfo,affiliation,authorrunning,paperdetails,acks,email},
  moretexcs=[3]{addbibresource,printbibliography,bibliography},%
}%
\lstset{%
  language={[programming]TeX},%
  keywordstyle=\firamedium,
  stringstyle=\color{RosyBrown},%
  texcsstyle=*{\color{Purple}\mdseries},%
  texcsstyle=*[2]{\color{Blue1}},%
  texcsstyle=*[3]{\color{ForestGreen}},%
  commentstyle={\color{FireBrick}},%
  escapechar=`,%
}
\newcommand*{\CTAN}[1]{%
  \href{http://ctan.org/tex-archive/#1}{\nolinkurl{CTAN:#1}}%
}


\begin{document}


\title{Writing Articles for The Art, Science, and Engineering of Programming}
\titlerunning{Writing for Programming} %optional, in case that the title is too long; the running title should fit into the top page column

\author[a]{Tobias Pape}
\authorinfo{is the author of this {LaTeX} class. Contact him at
  \email{tobias.pape@hpi.uni-potsdam.de}.}
\affiliation[a]{Hasso Plattner Institute, University of Potsdam, Germany}
\author{Cristina V. Lopes}
\authorinfo{is associate editor for the first two issues of The Art, Science,
  and Engineering of Programming. Contact her at \email{lopes@ics.uci.edu}.}
\affiliation{University of California, Irvine, USA}
\author[a]{Robert Hirschfeld}
\authorinfo{is chair of the AOSA steering committee. The Art, Science,
  and Engineering of Programming is published by AOSA. Contact Robert at \email{hirschfeld@hpi.uni-potsdam.de}.}

% \authorrunning{T. Pape, C. Lopes, R. Hirschfeld} % Optional, for long author lists

\keywords{programming journal, paper formatting, submission preparation} % please provide 1--5 keywords


%%%%%%%%%%%%%%%%%%
%% These data must be filled for your submission. May be left out in final version
\paperdetails{
  area={Social Coding},
  perspective=art,
}
%%%%%%%%%%%%%%%%%%

%%%%%%%%%%%%%%%%%%
%% These data are provided by the editors. May be left out on submission
\paperdetails{
  submitted=2016-08-10,
  published=2016-10-11,
  year=2016,
  volume=1,
  issue=1,
  articlenumber=1,
}
%%%%%%%%%%%%%%%%%%

%%%%%%%%%%%%%%%%%%%%%%%%%%%%%
% Please go to https://dl.acm.org/ccs/ccs.cfm and generate your Classification
% System [view CCS TeX Code] stanz and copy _all of it_ to this place.
%% From HERE
\begin{CCSXML}
<ccs2012>
<concept>
<concept_id>10002944.10011122.10003459</concept_id>
<concept_desc>General and reference~Computing standards, RFCs and guidelines</concept_desc>
<concept_significance>300</concept_significance>
</concept>
<concept>
<concept_id>10010405.10010476.10010477</concept_id>
<concept_desc>Applied computing~Publishing</concept_desc>
<concept_significance>300</concept_significance>
</concept>
</ccs2012>
\end{CCSXML}

\ccsdesc[300]{General and reference~Computing standards, RFCs and guidelines}
\ccsdesc[500]{Applied computing~Publishing}

% To HERE
%%%%%%%%%%%%%%%%%%%%%%%

\maketitle

% Please always include the abstract.
\begin{abstract}
  The Art, Science, and Engineering of Programming is a new journal created
  with the goal of placing the wonderful art of programming in the map of
  scholarly works. Many academic journals and conferences exist that publish
  research related to programming, starting with programming languages,
  software engineering, and expanding to the whole Computer Science field. Yet,
  many of us feel that, as the field of Computer Science expanded, programming,
  in itself, has been neglected to a secondary role not worthy of scholarly
  attention. That is a serious gap, as much of the progress in Computer Science
  lies on the basis of computer programs, the people who write them, and the
  concepts and tools available to them to express computational tasks.

  The Art, Science, and Engineering of Programming aims at closing this gap by
  focusing primarily on programming: the art itself (programming styles,
  pearls, models, languages), the emerging science of understanding what works
  and what doesn’t work in general and in specific contexts, as well as more
  established engineering and mathematical perspectives.
  
  This is an example of and a guide to writing articles for The Art, Science,
  and Engineering of Programming.
 \end{abstract}

\section{A Guided Tour}
\label{sec:guided-tour}

The content of a paper is its most important ingredient. That being said,
papers for The Art, Science, and Engineering of Programming should adhere to a
common style, which is achieved with the \lstinline|programming| {LaTeX}
class. You are to use this class as outlined here when submitting articles to
the journal. This document acts as an example of what this can look like and a
short references for all the required and optional parts.

\subsection{Document Parts}
\label{sec:document-parts}


\begin{lstlisting}[float,label=lst:tour,caption={A very minimal article.}]
\documentclass[english,submission]{programming}`\label{l:doc}`
\usepackage[backend=biber]{biblatex} % Use Biblatex `\label{l:bib}`
\addbibresource{example.bib}

\begin{document}

\paperdetails{`\label{l:details}`
  perspective=engineering,
  area={General-purpose programming}
}

\title{The importance of why and how to do work}`\label{l:meta}`
\author{Anna Author}
\affiliation{The Unseen University, Ankh-Morpork}
\author{Bert Betatester}
\affiliation{Carolingian Minuschool Academy, Bodoni, San Serriffe}

\keywords{paper, showcase, lorem ipsum}`\label{l:keywords}`

\begin{CCSXML}`\label{l:ccs:start}`
<ccs2012>
<concept>
<concept_id>10002944.10011122.10003459</concept_id>
<concept_desc>General and reference~Computing standards, RFCs and guidelines</concept_desc>
<concept_significance>300</concept_significance>
</concept>
</ccs2012>
\end{CCSXML}

\ccsdesc[300]{General and reference~Computing standards, RFCs and guidelines}`\label{l:ccs:end}`

\maketitle

\begin{abstract}
  This paper shows...
\end{abstract}

\section{Introduction}
The art of computer programming~\cite{DBLP:journals/cacm/Knuth74}...
\section{...}
...
\acks`\label{l:acks}`
I want to thank ....

\printbibliography
\end{document}
\end{lstlisting}

{\lstset{language=}%
  We have included a most minimal article in \autoref{lst:tour} that you can
  adapt to the concrete work to be submitted\footnote{Please note that all
    article source must be \textsc{utf-8} encoded, no exceptions.}. First, you
  have to select the \lstinline|programming| class as in \autoref{l:doc}. The
  parameter \lstinline|english| is for language selection (optional, but
  recommended). The other parameter, \lstinline|submission|, is crucial for the
  reviewing process and must be included for submission. More information on
  document parameters can be found in \autoref{sec:class-options}. After that,
  wce recommend that you load the \lstinline|biblatex| package. While it is
  perfectly fine to use plain {BibTeX}, we recommend
  \lstinline|biblatex|, as it has native Unicode support and can handle
  \textsc{url}s and \textsc{doi}s better. In both cases, we have preloaded a
  bibliography style.

At that point, you can load any {LaTeX} package you need. Note that we
already load some important packages with the \lstinline|programming| class and
that there are certain packages that are not recommended or incompatible with
this class. Please refer to \autoref{sec:pack-cons} for information on which
packages are recommended.
After that, you start the actual document with \lstinline|\begin{document}|.

  You have to provide details to your submission that will be used during the
  review process. For that, use the \lstinline|paperdetails| command and
  specify \emph{perspective} and \emph{area} of your submission. For
  perspective, choose form one of the \emph{Art}, \emph{theoretical} or
  \emph{empirical} \emph{Science}, or \emph{Engineering} and provide it as
  argument to the \lstinline|perspective| keyword; they can be found in
  \autoref{sec:perspective}. For area, specify an argument to the
  \lstinline|area| keyword (in braces). Find a list of suggested areas in
  \autoref{sec:area-submission}.

We need you to provide certain metadata which you can do with the following
commands, as form \autoref{l:meta}. You can specify the title and maybe the subtitle with
\lstinline|\title| (and \lstinline|\subtitle|, respectively). Then please
list all authors. For that, use one \lstinline|\author| command per author
and give an affiliation for each with \lstinline|\affiliation|. When several
consecutive authors share the same affiliation, you should only use one
\lstinline|\affiliation| command after the last of them. More information can
be found in \autoref{sec:document-metadata}. 

Please include one to five keywords, separated by commas, that are important to
your work with the \lstinline|\keywords| command as in \autoref{l:keywords}. We
kindly ask you to classify your work using the \emph{ACM Computing
  Classification System}\footnote{\url{https://dl.acm.org/ccs/ccs.cfm}.}.
Please visit the ACM's website and generate the {TeX} code that matches
your classification. The code should consist of a \lstinline|CCSXML|
environment and several \lstinline|\ccsdesc| commands, as in lines
\ref{l:ccs:start} to \ref{l:ccs:end}.

With \lstinline|\maketitle| and the \lstinline|abstract| environment, you can
produce the article's title page. Then you can start with your content. If
applicable, you can put acknowledgements at the end after an \lstinline|\acks|
command, as in \autoref{l:acks}.
}

This should cover the basic usage.

\subsection{Running {LaTeX}}
\label{sec:running}

This class has been tested with the popular {LaTeX} programs available.
It should work with {pdfLaTeX}, {LuaLaTeX}, and {XeLaTeX}.
We suggest that you use {LuaLaTeX}, which is included in all major
{TeX} system installations, has a good Unicode support, and is well portable.
Please do not use the LaTeX/dvips combination to produce output files.
To get all metadata right, several LaTeX runs will be necessary. Hence, a
typical run looks like this:
\begin{lstlisting}[language=]
lualatex myarticle
biber myarticle
lualatex myarticle
lualatex myarticle
\end{lstlisting}

You should replace \lstinline|lualatex| with the LaTeX program of your choice,
and \lstinline|biber| with \lstinline|bibtex| if you do not want to use
Biblatex (or use Biblatex's BibTeX backend).

Your \textsc{pdf} file for your article should then be ready for submission.

\section{Submission Classification}

Almost anything about programming is in scope for The Art, Science, and
Engineering of Programming. However, you should classify your article according
to the following aspects.




\subsection{Perspective}
\label{sec:perspective}

We accept descriptions of work under different perspectives:

\emph{The Art.} This perspective is about knowledge and technical skills acquired through
practice and personal experiences. Examples include libraries, frameworks,
languages, APIs, programming models and styles, programming pearls, and essays
about programming.

\emph{Science (theoretical).} This perspective is about knowledge and technical skills acquired
through mathematical formalisms. Examples include formal programming models and
proofs.

\emph{Science (empirical).} This perspective is about knowledge and technical skills acquired
through experiments and systematic observations. Examples include user studies
and programming-related data mining.

\emph{Engineering.} This perspective is about knowledge and technical skills acquired through
designing and building large systems and through calculated application of
principles in building those systems. Examples include measurements of
artifacts’ properties, development processes and tools, and quality assurance
methods.

\bigskip
To classify your work as any of the above perspectives, please set the
\lstinline|perspective| keyword of  the \lstinline|\paperdetails| command to
one of the following values:

\begin{description}
\item[art] for submissions concerning \emph{The Art }
  (\textsf{theart} is a proper alias);
\item[sciencetheoretical] For submissions concerning \emph{Science
    (theoretical)}, with aliases \textsf{sciencetheoretical},
  \textsf{theoreticalscience}, \textsf{theoretical},
  \textsf{science-theoretical}, and \textsf{theoretical-science};
\item[scienceempirical] for submissions concerning \emph{Science (empirical)},
  aliases are \textsf{scienceempirical}, \textsf{empiricalscience},
  \textsf{empirical}, \textsf{science-empirical}, and
  \textsf{empirical-science}; and
\item[engineering] for submissions concerning \emph{Engineering}.
\end{description}

\subsection{Area of Submission}
\label{sec:area-submission}

Independent of the type of work, the journal accepts submissions covering
several expertise areas. Expertise areas include, but are not limited to:

\begin{itemize}
\item General-purpose programming
\item Distributed systems programming
\item Parallel and multi-core programming
\item Graphics and GPU programming
\item Security programming
\item User interface programming
\item Database programming
\item Visual and live programming
\item Data mining and machine learning programming, and for programming
\item Interpreters, virtual machines and compilers
\item Modeling and modularity
\item Testing and debugging
\item Program verification
\item Programming education
\item Programming environments
\item Social coding
\end{itemize}

To specify the area of expertise for your work, please set the
\lstinline|area| keyword of  the \lstinline|\paperdetails| command to
one of the preceding suggested areas, or provide your own. Please use curly
braces \lstinline|{}| around the value.

\section{Typographical and Technical Aspects}

We have carefully chosen the presentation for articles published in The Art, Science, and
Engineering of Programming to fit on-screen reading and occasional printouts.
It is important to maintain a uniform look of all articles appearing in the
journal. We kindly ask you therefore to \emph{not} change the presentation. Most
importantly,
%
\begin{itemize}
\item please do not change the font or the font size for that matter. The text
  font remains fixed as \emph{Charter} (in the XCharter variant) and emphasis
  font is \emph{Fira Sans}. The only exception is the typewriter font, where
  you can choose from three variants (please refer to
  \autoref{sec:typewriter-fonts}). Moreover;
\item please do not change the margins or the line spacing;
\item please refrain from using the \lstinline|\sloppy| command, especially for
  the whole document;
\item please avoid the use of the \lstinline|[H]| or \lstinline|[h!]| modifiers
  for figures;
\item please use a consistent formatting  (for example, use the
  \lstinline|siunitx| package~\cite{siunitx}.)
\end{itemize}
%
This class has been tested on several LaTeX systems. We suggest that you use an
up-to-date system, for example
TeX\,Live\footnote{\url{https://www.tug.org/texlive/}.}, which is available for
Linux, Windows, and Apple systems, among others. We tested with the 2015 and
2016 systems and will make sure that it will work with upcoming systems, too.
Please first upgrade your system if problems with this class should occur.
Regarding the LaTeX system, please keep in mind to
%
\begin{itemize}
\item use LaTeX packages only if required, as fewer packages lower the
  chance of conflicts (refer to \autoref{sec:pack-cons} for details);
\item keep the use of custom macros low to not interfere with provided
  packages and use LaTeX commands to do so (for example, \lstinline|\newcommand| and
  similar instead of TeX's \lstinline|\def|);
\item watch out for unsuitable line- and pagebreaks;
\item when using pixel graphics, provide at least 300\,dpi imagery,
  \textsc{pdf} files preferred;
\item minimize the use of LaTeX comments, especially via block constructs;
\item consult the \emph{l2tabu} documentation~\cite{l2tabu} about outdated packages;
\item provide \textsc{doi}s for all your references, if possible.
\end{itemize}
%
We suggest the use of a spellchecker to avoid typos.

\section{Detailed Reference}

Several aspects of this class as outlined above provide for some fine tuning.

\subsection{Class Options}
\label{sec:class-options}

As explained in~\autoref{sec:guided-tour}, submissions to the journal must use
the \lstinline|submission| package option. For the accepted final, to-be-published
version, this option should be omitted or replaced by  \lstinline|crc| for
“Camera-ready Copy”. Actually, both are shortcuts for the
\lstinline|phase| keyword, which could be used, instead. 
Hence, the following statements in either column are equivalent:\par\noindent
{\lstset{texcsstyle=*{\color{Purple}\mdseries\smaller}}
\begin{minipage}[t]{.45\linewidth}
\emph{Final phase}
\begin{lstlisting}
\documentclass[phase=final]{programming}
\documentclass[crc]{programming}
\documentclass{programming}
\end{lstlisting}
\end{minipage}%
\hfill%
\begin{minipage}[t]{.5\linewidth}
\emph{Submission phase}
\begin{lstlisting}[numbers= none]
\documentclass[phase=submission]{programming}
\documentclass[submission]{programming}
``
\end{lstlisting}
\end{minipage}}

\paragraph{Code Fonts}
\label{sec:typewriter-fonts}

The standard font for code, as selected with \lstinline|\texttt|,
\lstinline|\ttfamily|, or a listing, is set to be the same as the sans serif
font, \emph{Fira Sans}. This matches the rest of the document style very well.
However, it may be strictly necessary to have a non-proportional (or
monospaced) font for the typewriter style. We therefore provide a class option \lstinline|code|
to change the typewriter font. 
It takes one the following values:
\begin{description}
\item[code=sf] Sets the code font to the sans serif font (Fira Sans). This is the
  default. Aliases for this option are \lstinline|code=sans| and
  \lstinline|code=sansserif|.
\item[code=tt] Sets the code font to Fira Sans' monospaced equivalent, Fira Mono. Aliases for this option are \lstinline|code=mono| and
  \lstinline|code=monospace|. This is the preferred choice if monospaced
  typewriter font is necessary. Note that this font does not have italics.
\end{description}

\paragraph{Other Options}

All other options give in the \lstinline|\documentclass| command are not
processed directly by the class but rather passed on to other packages as
``global options''. For example, the \lstinline|english| option from the
example above does not influence the \lstinline|programming| class directly,
but rather will be picked up by Babel~\cite{babel} to set up language options.

\subsection{Document Metadata}
\label{sec:document-metadata}

The presentation of the document metadata can be fine tuned, as well.
Besides the \lstinline|\title| and \lstinline|\subtitle| commands, there is the
\lstinline|\titlerunning| command to specify a shorter title when the normal
title does not fit into the page header. Similarly, the
\lstinline|\authorrunning| command can be used to specify the presentation of
the authors in the header.

Sometime listing the affiliations of the authors can be tricky when they are
shared. An optional argument to the \lstinline|\author| and
\lstinline|\affiliation| command can be used to associate authors and
affiliations explicitly. Four variants for affiliations are outlined in the
beginning of \autoref{lst:affiliation}.

For the final publication, we ask you to provide a short information of each
author, typically a very short biography, and a photo. You can use the
\lstinline|\authorinfo| command right after an \lstinline|\author| command to
associate that author information with an author. See \autoref{l:authorinfo} in
\autoref{lst:affiliation} for example. The photo is given as optional argument
as file name in brackets (without file extension). Omitting the photo will
produce a blank space next to the author information. The actual information
follows as an argument. Note that the author's \emph{name} is prepended to the
information automatically and is not to be repeated.

An author's email address can be given in either the affiliation or author
information block. Please use the \lstinline|\email| command for that purpose.

\begin{lstlisting}[float,label=lst:affiliation,caption={Four
ways to specify affiliations\,/\,Author information example\,/\,Sample publication paper details}]
% Independent affiliations 
\author{Anna Author}
\affiliation{The Unseen University, Ankh-Morpork}
\author{Bert Betatester}
\affiliation{Carolingian Minuschool Academy, Bodoni, San Serriffe}

% Shared affiliations 
\author{Anna Author}
\author{Bert Betatester}
\affiliation{Carolingian Minuschool Academy, Bodoni, San Serriffe}

% Explicit affiliations 
\author[a]{Anna Author}
\author[b]{Bert Betatester}
\author[a]{Claire Cadence}
\affiliation[a]{The Unseen University, Ankh-Morpork}
\affiliation[b]{Carolingian Minuschool Academy, Bodoni, San Serriffe}

% Mixed affiliations 
\author[a]{Anna Author}
\author[a,b]{Bert Betatester}
\affiliation[a]{The Unseen University, Ankh-Morpork}
\affiliation[b]{Carolingian Minuschool Academy, Bodoni, San Serriffe}

% Author information`\label{l:authorinfo}`
\author{Anna Author}
\authorinfo[anna]{is a researcher at the Unseen University in Ankh-Morpork and
a Ph\,D student to Rincewind. Contact her at
\email{anna.the.author@univ-unseen.edu.pratchett}}  
\affiliation[a]{The Unseen University, Ankh-Morpork}


%%%%%%%%%% Sample publication paper details
\paperdetails{
  submitted=2017-02-18,
  published=2017-08-01,
  year=2017,
  volume=2,
  issue=1,
  articlenumber=4,
}
\end{lstlisting}

\subsection{Paper Details}

Besides specifying the perspective and area of the submission, the
\lstinline|\paperdetails| command is also used to provide publication
information for the final, camera-ready-copy phase (class option
\lstinline|phase=final|). In this case, perspective and area are removed from
the title page and the Digital Object Identifier (\textsc{doi}), as well as the
dates of submission and publications are presented instead. The information
necessary for this is provided \emph{by the publisher} after acceptance. It
includes the following information:
%
\begin{description}
\item[year] The year of the articles publication.
\item[volume] The volume the article appears in.
\item[issue] The issue the article appears in.
\item[articlenumber] The number of the article within the volume.
\item[submitted] The date the article was submitted to (and received by) the
  publisher. Please use an \textsc{iso} date (\textsc{yyyy-mm-dd}).
\item[publisher] The date the article was published and made public by the
  publisher. Please use an \textsc{iso} date (\textsc{yyyy-mm-dd}).
\end{description}
%
See the end of \autoref{lst:affiliation} for an example.

\subsection{Figures, Tables, Listings}

Sometimes it is necessary to pay special attention to the \emph{floating
  elements} of your document, that is figures, tables, listings, and similar. 
While it is tempting to force those elements to the specific position where
they are written down, this is not always the best choice, especially since it
can interrupt the reading flow. Therefore please refrain form using the
\lstinline|[h!]| and  \lstinline|[H]| specifiers for such elements. If you
\emph{must} have the element at the current position, it maybe is not a
\emph{floating element}, in the first place. It is perfectly fine to use
graphics \emph{without} a \lstinline|\begin{figrue}...\end{figure}| environment
or use a \lstinline|\begin{tabular}...\end{tabular}| \emph{without} an
enclosing \lstinline|\begin{table}...\end{table}|.


\begin{figure}
  \centering
  \setlength{\unitlength}{.01in}%{.025in}
  \begin{picture}(200,75)
    \put(0,25){\vector(1,0){200}}
    \put(25,0){\vector(0,1){75}}
    \put(75,22){\line(0,1){6}}
    \put(125,22){\line(0,1){6}}
    \put(22,50){\line(1,0){6}}
    \thicklines
    \put(25,25){\line(1,0){50}}
    \put(75,50){\line(1,0){50}}
    \put(125,25){\line(1,0){72}}
    \put(17,50){\makebox(0,0){$1$}}
    \put(75,13){\makebox(0,0)[b]{$\pi$}}
    \put(125,13){\makebox(0,0)[b]{$2\pi$}}
    \put(195,13){\makebox(0,0)[b]{$t$}}
    \put(175,60){\makebox(0,0){$g(t)$}}
  \end{picture}
  \caption{A test figure}
  \label{fig:test}
\end{figure}


The caption position of a floating element should match its content. For
figures, graphs, graphics, or similar, please us captions \emph{below} the
imagery, as in \autoref{fig:test}. For tables, listings, algorithms, or similar list-like content, please
use captions \emph{above} its contents, as those elements are typically read
from top to bottom. Find examples in the listings already presented and
\autoref{tbl:things}.

For tabular content, we suggest to avoid vertical lines altogether and keep
horizontal lines to a minimum. Refer to the \lstinline|booktabs| package
documentation for more information~\cite{booktabs}. Also, the
\lstinline|threeparttable| package is valuable if you need footnotes in your
table~\cite{threeparttable}.

\begin{table}
  \centering
  \begin{threeparttable}
    \caption{Differences between things projected and things achieved}
    \label{tbl:things}
    \small
    \begin{tabular}{>{}p{.4\linewidth}@{}cSs}
      \toprule
      Part           & done & {Value}   & Unit                      \\
      \midrule
      Title          & yes  &    2.3456 & \dB                       \\
      Abstract       & no   &           &                           \\
      Intro          & yes  &       e3  & \kilo\hertz               \\
      \midrule
      \multicolumn{4}{c}{Rest is not entirely true}                 \\
      \midrule
      Context        & yes  &   90.473  & \percent                  \\
      Problem        & no\tnote{a} &    &                           \\
      Solution       & yes  & 5642.5    & \mega\byte                \\
      Implementation & yes  &    1.2e-3 & \metre\squared\per\second \\
      Evaluation     & no   &           &                           \\
      Related Work   & no   &           &                           \\
      Conclusion     & yes  &    4955.3 & \per\kilogram             \\
      \bottomrule
    \end{tabular}
    \begin{tablenotes}
    \item [a] Just a few things missing
    \end{tablenotes}
  \end{threeparttable}
\end{table}


\subsection{Bibliography}

As outlined, the \lstinline|programming| class supports both plain BibTeX as
well as the newer Biblatex bibliography package to present your references.
Please note that you may not change the bibliography style, it is pre-selected
by the class. 

Programming uses a simple, numbered bibliography scheme with all references
sorted alphabetically, which should look like this:
\cite{DBLP:journals/cacm/Knuth74}. Note that the reference has the \textsc{doi}
included. We would like to encourage you to always provide the \textsc{doi}s of
the works you cite, if possible. However, please make sure in your bibliography
file, that the \lstinline|doi| entry a plain entry without a ``resolver'' (for
example, \textsf{http://dx.doi.org/}) prepended. A typical example for an entry
format should look like in \autoref{lst:bib}. Please shorten neither names nor
journal articles, if possible.

The bibliography commands for your article should look like this (for Biblatex
and plain BibTeX, respectively):\par\noindent
\begin{minipage}[t]{.45\linewidth}
\emph{Biblatex}
\begin{lstlisting}
\documentclass{programming}
\usepackage[backend=biber]{biblatex}
\addbibresource{example.bib}
\begin{document}
...
\printbibliography
\end{document}
\end{lstlisting}
\end{minipage}%
\hfill%
\begin{minipage}[t]{.45\linewidth}
\emph{Plain  Bibtex}
\begin{lstlisting}[numbers= none]
\documentclass{programming}


\begin{document}
...
\bibliography{example}
\end{document}
\end{lstlisting}
\end{minipage}


\begin{lstlisting}[float,label=lst:bib,language=,caption={A typical BibTeX entry for \cite{DBLP:journals/cacm/Knuth74}}]
@article{DBLP:journals/cacm/Knuth74,
  Author = {Donald E. Knuth},
  Doi = {10.1145/361604.361612},
  Journal = {Communications of the {ACM}},
  Number = {12},
  Pages = {667--673},
  Title = {{Computer Programming as an Art}},
  Volume = {17},
  Year = {1974}}
\end{lstlisting}


\subsection{{LaTeX} Packages Considerations }
\label{sec:pack-cons}

The following packages are automatically loaded by the \lstinline|programming|
class and need not to be loaded manually.

\begin{multicols}{3}
{\color{DarkGreen}
\begin{itemize}
\item \path{accsupp}
\item \path{amsmath}
\item \path{amstext}
\item \path{array}
\item \path{atbegshi}
\item \path{babel}
\item \path{booktabs}
\item \path{calc}
\item \path{caption}
\item \path{colortbl}
\item \path{comment}
\item \path{csquotes}
\item \path{doclicense}
\item \path{expl3}
\item \path{FiraSans}
\item \path{fnpct}
\item \path{fontspec}\footnote{LuaLaTeX\,/\,XeLaTeX only}
\item \path{fontenc}
\item \path{gettitlestring}
\item \path{graphics}
\item \path{graphicx}
\item \path{grfext}
\item \path{hypcap}
\item \path{hyperref}
\item \path{hyperxmp}
\item \path{listings}
\item \path{mathdesign}
\item \path{microtype}
\item \path{morewrites}
\item \path{multirow}
\item \path{relsize}
\item \path{rotating}
\item \path{siunitx}
\item \path{subcaption}
\item \path{tabularx}
\item \path{textcase}
\item \path{textcomp}
\item \path{threeparttable}
\item \path{typearea}
\item \path{url}
\item \path{verbatim}
\item \path{xcolor}
\item \path{xparse}
\item \path{xspace}
\item \path{xunicode}
\end{itemize}}
\end{multicols}

\par\noindent The following packages \emph{must not} be loaded.

\begin{multicols}{3}
{\color{DarkRed}
\begin{itemize}
\item \path{SIstyle}
\item \path{SIunits}
\item \path{a4wide}
\item \path{a4}
\item \path{aecompl}
\item \path{ae}
\item \path{caption2}
\item \path{courier}
\item \path{doublespace}
\item \path{epsfig}
\item \path{epsf}
\item \path{euler}
\item \path{fancyhdr}
\item \path{fancyheadings}
\item \path{fourier}
\item \path{geometry}
\item \path{glossary}
\item \path{helvet}
\item \path{isolatin}
\item \path{mathpple}
\item \path{mathptmx}
\item \path{mathptm}
\item \path{newtxmath}
\item \path{newtxtext}
\item \path{palatino}
\item \path{psfig}
\item \path{pslatex}
\item \path{scrpage}
\item \path{subfigure}
\item \path{subfig}
\item \path{t1enc}
\item \path{times}
\item \path{umlaut}
\item \path{utopia}
\item \path{zefonts}
\end{itemize}}
\end{multicols}

Additionally, any package that changes the font must not be loaded.

\acks
We want to thank you for using this class to prepare articles for The Art, Science, and Engineering of Programming.
\printbibliography

\appendix
\section{A Famous Filler Text}

Lorem ipsum dolor sit amet, consectetur adipiscing elit, sed do eiusmod tempor
incididunt ut labore et dolore magna aliqua. Ut enim ad minim veniam, quis
nostrud exercitation ullamco laboris nisi ut aliquip ex ea commodo consequat.
Duis aute irure dolor in reprehenderit in voluptate velit esse cillum dolore eu
fugiat nulla pariatur. Excepteur sint occaecat cupidatat non proident, sunt in
culpa qui officia deserunt mollit anim id est laborum. 

\end{document}

% Local Variables:
% TeX-engine: luatex
% End:
